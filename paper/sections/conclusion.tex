The General Purpose Graphics Processing Unit is becoming increasingly more relevant now that disciplines outside of games and multimedia are finding classes of problems that it can efficiently solve. However, as GPGPUs are currently being used as an additional component to the computer, they are contributing to the power wall which is slowing performance innovation. Also, as the classes of problems that GPGPUs can solve grows, so does the amount of data that they process. Transferring all of that data to an external card was proving to be a bottleneck on its own. 

Integrating GPGPUs into the CPU itself was the first step in trying to solve this problem. This presented its own class of challenges and opportunities for the programmer and system architect. This also started to alleviate stress on the power wall by allowing the GPGPU to utilize the memory of the CPU and the cooling infrastructure. However problems still existed as the GPGPU was being treated as a second class chip when compared to the CPU. It still was controlled and was issued instructions from the CPU. Also, data was being transferred to the GPGPU over a bus tying up the CPU.

As GPGPUs move into the future, heterogenous architectures are starting be created. This means that the GPGPU is directly included as a core on the CPU itself. Allowing the GPGPU to be accessed and run on its own provides challenges to systems architects. This paper identified two classes of GPGPU programs that could be run. The first was a stand alone program that had no interaction from the CPU. The second represented a more complicated GPGPU/CPU mixed program. These will typically be more common. The paper then identified solutions to these problems. 

The first case was solved with compiler and operating systems extensions supporting the GPGPU as an independently addressable core. The second more common case involved an instruction set modification allowing the program to pass both GPGPU instructions with CPU instructions. This allows the GPGPU to run without the program itself initiating the work, as a true additional core. 